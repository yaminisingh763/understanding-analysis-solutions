\documentclass[12pt]{article}

\usepackage[a4paper, margin=1in]{geometry}

\usepackage{amsmath, amsfonts, amssymb, amsthm}
\usepackage{mathrsfs}

\usepackage{graphicx}
\usepackage{float}
\usepackage{xcolor}
\usepackage{tcolorbox}
\usepackage{enumitem}

\usepackage{tikz}
\usepackage{stackengine}
\usepackage{scalerel}

\usepackage{hyperref}

\setlength{\parindent}{0pt}

\newcommand{\mustachebrace}[2]{%
  \underbrace{#1}_{\text{\begin{tikzpicture}[baseline=(current bounding box.center)]
    \node[anchor=base] (a) at (0,0) {\scaleto{\text{#2}}{4pt}};
  \end{tikzpicture}}}
}

\title{\textbf{Solutions to \\ \textit{Understanding Analysis} by Stephen Abbott}}
\author{Yamini Singh}
\date{}

\begin{document}

\maketitle

\section*{Exercises}

\textbf{Exercise 1.4.1.} Recall that $I$ stands for the set of irrational numbers.

\begin{enumerate}[(a)]
    \item Show that if $a,b \in \mathbb{Q}$, then $ab$ and $a + b$ are elements of $\mathbb{Q}$ as well.

    \textbf{Solution.}

\textit{Proof.}

Since $a,b \in \mathbb{Q}$,
\[
a = \frac{p}{q}, \quad b = \frac{m}{n},
\]
for some $p,m \in \mathbb{Z}$ and $q,n \in \mathbb{N}$ with $q \neq 0$ and $n \neq 0$.

Then
\[
ab = \frac{pm}{qn}.
\]

Since $pm \in \mathbb{Z}$ and $qn \in \mathbb{N}$ with $qn \neq 0$,
\[
\Rightarrow ab \in \mathbb{Q}.
\]

Also,
\[
a+b = \frac{p}{q} + \frac{m}{n} = \frac{pn + qm}{qn}.
\]

Since $pn + qm \in \mathbb{Z}$ and $qn \in \mathbb{N}$ with $qn \neq 0$,
\[
\Rightarrow a+b \in \mathbb{Q}.
\]

Thus, $\mathbb{Q}$ is closed under addition and multiplication.
\qed
    \item Show that if $a \in \mathbb{Q}$ and $t \in I$, then $a + t \in I$ and $at \in I$ as long as $a \ne 0$.
    \item Part (a) can be summarized by saying that $\mathbb{Q}$ is closed under addition and multiplication. 
    Is $I$ closed under addition and multiplication? Given two irrational numbers $s$ and $t$, what can we say about $s+t$ and $st$?
\end{enumerate}

\vspace{0.2cm}

\textbf{Exercise 1.4.2.} Let $A \subseteq \mathbb{R}$ be nonempty and bounded above, and let $s \in \mathbb{R}$ have the property that for all $n \in \mathbb{N}$, $s + \frac{1}{n}$ is an upper bound for $A$ and $s - \frac{1}{n}$ is not an upper bound for $A$. Show $s = \sup A$.

\textbf{Solution.}

\textit{Step 1: $s$ is an upper bound for $A$.}

Since
\[
s + \frac{1}{n} \text{ is an upper bound for } A \quad \text{for all } n \in \mathbb{N},
\]
we have
\[
x \le s + \frac{1}{n} \quad \text{for all } x \in A \text{ and all } n \in \mathbb{N}.
\]

Letting $n \to \infty$, we obtain
\[
x \le s \quad \text{for all } x \in A.
\]

\[
\Rightarrow s \text{ is an upper bound for } A.
\]

\textit{Step 2: $s$ is the least upper bound of $A$.}

Since
\[
s - \frac{1}{n} \text{ is not an upper bound for } A \quad \text{for all } n \in \mathbb{N},
\]
it follows that for each $n \in \mathbb{N}$, there exists an element $x_n \in A$ such that
\[
s - \frac{1}{n} < x_n.
\]

Now let $\varepsilon > 0$ be given.

By the Archimedean Property, there exists $m \in \mathbb{N}$ such that
\[
\frac{1}{m} < \varepsilon.
\]

Since $s - \frac{1}{m}$ is not an upper bound for $A$, there exists $x \in A$ such that
\[
s - \frac{1}{m} < x.
\]

Thus,
\[
s - \varepsilon < s - \frac{1}{m} < x \le s.
\]

\[
\Rightarrow \text{for every } \varepsilon > 0, \text{ there exists } x \in A \text{ such that } s - \varepsilon < x \le s.
\]

Hence, no number smaller than $s$ can be an upper bound for $A$.

\[
\Rightarrow s \text{ is the least upper bound of } A.
\]

Therefore,
\[
\boxed{s = \sup A.}
\]

\vspace{0.2cm}

\textbf{Exercise 1.4.3.} Prove that $\bigcap_{n=1}^{\infty} (0, 1/n) = \varnothing$. Notice that this demonstrates that the intervals in the Nested Interval Property must be closed for the conclusion of the theorem to hold.

\textbf{Proof.}
Suppose, for contradiction, that
\[
\bigcap_{n=1}^{\infty} (0,1/n)\neq\varnothing.
\]
Then there exists some $x\in\mathbb{R}$ such that
\[
x\in (0,1/n)\quad \text{for all } n\in\mathbb{N}.
\]

Define
\[
I_n=(0,1/n), \quad n\in\mathbb{N}.
\]
Then
\[
x\in I_n \ \forall n \in \mathbb{N}
\;\Rightarrow\;
0<x<\frac{1}{n} \ \forall n\in\mathbb{N}.
\]

Thus,
\[
x \le \frac{1}{n} \quad \forall n\in\mathbb{N},
\]
which implies that $x$ is a lower bound of the set
\[
\left\{\frac{1}{n}: n\in\mathbb{N}\right\}.
\]

But
\[
\inf\left\{\frac{1}{n}: n\in\mathbb{N}\right\}=0,
\]
since for every $\varepsilon>0$, there exists $n\in\mathbb{N}$ such that
\[
\frac{1}{n}<\varepsilon
\]
(by the Archimedean Property).

Hence,
\[
x \le 0.
\]
This contradicts the fact that $x>0$.

Therefore, no such $x$ exists, and we conclude
\[
\bigcap_{n=1}^{\infty} (0,1/n)=\varnothing.
\]
\qed

\vspace{0.2cm}

\textbf{Exercise 1.4.4.} Let $a < b$ be real numbers and consider the set $T = \mathbb{Q} \cap [a,b]$. Show $\sup T = b$.

\textbf{Solution.}

\textit{Step 1: $b$ is an upper bound for $T$.}

For all $x \in [a,b]$, we have
\[
x \le b.
\]

Since $T \subseteq [a,b]$, it follows that
\[
x \le b \quad \text{for all } x \in T.
\]

\[
\Rightarrow b \text{ is an upper bound for } T.
\]

\textit{Step 2: $b$ is the least upper bound of $T$.}

Let $\varepsilon > 0$ be given, with $\varepsilon \le b - a$.

Since the rational numbers are dense in $\mathbb{R}$, there exists a rational number $q \in \mathbb{Q}$ such that
\[
b - \varepsilon < q < b.
\]

Because $a \le b - \varepsilon < q < b \le b$, we have
\[
q \in [a,b].
\]

Hence,
\[
q \in \mathbb{Q} \cap [a,b] = T.
\]

Thus, for every $\varepsilon > 0$, there exists $q \in T$ such that
\[
b - \varepsilon < q \le b.
\]

\[
\Rightarrow \text{no number smaller than } b \text{ can be an upper bound for } T.
\]

Therefore,
\[
\boxed{\sup T = b.}
\]

\vspace{0.2cm}

\textbf{Exercise 1.4.5.} Using Exercise 1.4.1, supply a proof for Corollary 1.4.4 by considering the real numbers $a - \sqrt{2}$ and $b - \sqrt{2}$.

\textbf{Proof.}
Let $a<b$ be real numbers. Consider the real numbers
\[
a-\sqrt{2} \quad \text{and} \quad b-\sqrt{2}.
\]
Since $a<b$, we have
\[
a-\sqrt{2} < b-\sqrt{2}.
\]

Because $\mathbb{Q}$ is dense in $\mathbb{R}$, there exists a rational number
\[
r \in \mathbb{Q}
\]
such that
\[
a-\sqrt{2} < r < b-\sqrt{2}.
\]

Adding $\sqrt{2}$ throughout gives
\[
a < r+\sqrt{2} < b.
\]

Now, $r\in\mathbb{Q}$ and $\sqrt{2}\in I$, so by Exercise 1.4.1(b),
\[
r+\sqrt{2} \in I.
\]

Thus, there exists an irrational number $t=r+\sqrt{2}$ such that
\[
a<t<b.
\]
\qed

\vspace{0.2cm}

\textbf{Exercise 1.4.6.} Recall that a set $B$ is \emph{dense} in $\mathbb{R}$ if an element of $B$ can be found between any two real numbers $a < b$. Which of the following sets are dense in $\mathbb{R}$? 
Take $p \in \mathbb{Z}$ and $q \in \mathbb{N}$ in every case.
\begin{enumerate}[(a)]
    \item The set of all rational numbers $p/q$ with $q \le 10$.
    \item The set of all rational numbers $p/q$ with $q$ a power of $2$.
    \item The set of all rational numbers $p/q$ with $10|p| \ge q$.
\end{enumerate}

\vspace{0.2cm}

\textbf{Exercise 1.4.7.} Finish the proof of Theorem 1.4.5 by showing that the assumption $\alpha^2 > 2$ leads to a contradiction of the fact that $\alpha = \sup T$.

\vspace{0.2cm}

\textbf{Exercise 1.4.8.} Give an example of each or state that the request is impossible. When a request is impossible, provide a compelling argument for why this is the case.

\begin{enumerate}[(a)]
    \item Two sets $A$ and $B$ with $A \cap B = \varnothing$, $\sup A = \sup B$, 
    $A \not\subseteq A^c$ and $\sup B \not\in B$.

    \textbf{Example exists.}

Let
\[
A = \mathbb{Q} \cap (0,2), \qquad B = I \cap (0,2).
\]
Then
\[
A \cap B = \emptyset,
\]
because no number is both rational and irrational. Moreover,
\[
\sup A = \sup B = 2,
\]
and since $2 \notin (0,2)$,
\[
2 \notin A \quad \text{and} \quad 2 \notin B.
\]
    
    \item A sequence of nested open intervals $J_1 \supseteq J_2 \supseteq J_3 \supseteq \cdots$ with $\bigcap_{n=1}^{\infty} J_n$ nonempty but containing only a finite number of elements.

\textbf{Impossible.}
    
    \item A sequence of nested unbounded closed intervals $L_1 \supseteq L_2 \supseteq L_3 \supseteq \cdots$ with $\bigcap_{n=1}^{\infty} L_n = \varnothing$.
    (An unbounded closed interval has the form $[a,\infty) = \{x \in \mathbb{R} : x \ge a\}$.)

\textbf{Example exists.}

Define
\[
L_n = [n,\infty).
\]
Then
\[
L_1 \supseteq L_2 \supseteq L_3 \supseteq \cdots,
\]
each $L_n$ is closed and unbounded, and
\[
\bigcap_{n=1}^\infty L_n = \emptyset,
\]
since no real number is greater than or equal to every natural number.
    
    \item A sequence of closed bounded (not necessarily nested) intervals $I_1, I_2, I_3, \ldots$ with the property that $\bigcap_{n=1}^N I_n \ne \varnothing$ for all $N \in \mathbb{N}$, but $\bigcap_{n=1}^{\infty} I_n = \varnothing$.

\textbf{Impossible.}
\end{enumerate}

\end{document}